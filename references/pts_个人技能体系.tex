\documentclass[12pt]{article}
\usepackage{xeCJK} % 字符集
\setCJKmainfont{文鼎PL简报宋}
\usepackage{geometry} % 页面边距
\geometry{left = 4pt, right = 8pt, top = 1cm, bottom = 1.5cm}
\usepackage{color}
\pagecolor[rgb]{0.9, 0.99, 0.9}
\usepackage[bookmarks = true, colorlinks = true, linkcolor = blue]{hyperref}
\usepackage{amsmath} % 数学公式排版
\usepackage{unicode-math}
\begin{document}

\begin{center}
\quad\vspace{4cm}\\
\Huge{个人技能体系}\vspace{8pt}\\
\Large{Personal Technology System}\vspace{32pt}\\
\end{center}
\begin{flushright}
\large{——作者:文雄昌(FooFooDamon@GitHub)\qquad}
\end{flushright}

%各种命令、符号汇总:https://blog.csdn.net/garfielder007/article/details/51646604

\newpage

\renewcommand{\contentsname}{目录}
\begin{center}
\tableofcontents
\end{center}
\newpage

\section{微积分}

\subsection{导数}

\subsubsection{定义}

导数原名叫导函数(derived function),其意义是函数值的变化速率。\\
对于一个函数$y = f(x)$,其在某一点$(x, y)$的导数值等于\textbf{过该点的切线的斜率},数学表达式为:\\

\begin{equation*}
\begin{aligned}
f'(x) & = \frac{dy}{dx}\\
& = \lim_{\Delta x \to 0}\frac{\Delta y}{\Delta x}\\
& = \lim_{\Delta x \to 0}\frac{f(x + \Delta x) - f(x)}{\Delta x}
\end{aligned}
\end{equation*}

显然,对于以下两个最简单的函数,其导数分别为:\\
(1)$f(x) = C$(C为常数)\qquad $\to \qquad f'(x) = 0$\\
(2)$f(x) = x \qquad \to \qquad f'(x) = 1$

\subsubsection{可导的条件}

单调、连续

\subsubsection{四则运算的求导法则}

\paragraph{\textbf{\emph{\underline{和}}}}

\quad \\

若有:

\begin{equation*}
\begin{aligned}
h(x) = f(x) + g(x)
\end{aligned}
\end{equation*}

则有:

\begin{equation*}
\begin{aligned}
h'(x) = f'(x) + g'(x)
\end{aligned}
\end{equation*}

推导如下:\\

\begin{equation*}
\begin{aligned}
h'(x) & = \lim_{\Delta x \to 0}\frac{(f(x + \Delta x) + g(x + \Delta x)) - (f(x) + g(x))}{\Delta x}\\
& = \lim_{\Delta x \to 0}\frac{f(x + \Delta x) - f(x)}{\Delta x} + \lim_{\Delta x \to 0}\frac{g(x + \Delta x) - g(x)}{\Delta x}\\
& = f'(x) + g'(x)
\end{aligned}
\end{equation*}

\paragraph{\textbf{\emph{\underline{差}}}}

\quad \\

若有:

\begin{equation*}
\begin{aligned}
h(x) = f(x) - g(x)
\end{aligned}
\end{equation*}

则有:

\begin{equation*}
\begin{aligned}
h'(x) = f'(x) - g'(x)
\end{aligned}
\end{equation*}

推导同上。

\paragraph{\textbf{\emph{\underline{积}}}}

\quad \\

若有:

\begin{equation*}
\begin{aligned}
h(x) = f(x) \cdot g(x)
\end{aligned}
\end{equation*}

则有:

\begin{equation*}
\begin{aligned}
h'(x) = f'(x)g(x) + f(x)g'(x)
\end{aligned}
\end{equation*}

推导如下:\\

\begin{equation*}
\begin{aligned}
h'(x) & = \lim_{\Delta x \to 0}\frac{f(x + \Delta x) \cdot g(x + \Delta x) - f(x) \cdot g(x)}{\Delta x}\\
& = \lim_{\Delta x \to 0}\frac{f(x + \Delta x) \cdot g(x + \Delta x) + f(x) \cdot g(x + \Delta x) - f(x) \cdot g(x + \Delta x) - f(x) \cdot g(x)}{\Delta x}\\
& = g(x + \Delta x)\lim_{\Delta x \to 0}\frac{f(x + \Delta x) - f(x)}{\Delta x} + f(x)\lim_{\Delta x \to 0}\frac{g(x + \Delta x) - g(x)}{\Delta x}\\
& = f'(x)g(x) + f(x)g'(x)
\end{aligned}
\end{equation*}
\\
\paragraph{\textbf{\emph{\underline{商}}}}

\quad \\

若有:

\begin{equation*}
\begin{aligned}
h(x) = \frac{f(x)}{g(x)}
\end{aligned}
\end{equation*}

则有:

\begin{equation*}
\begin{aligned}
h'(x) = \frac{f'(x)g(x) - f(x)g'(x)}{g^2(x)}
\end{aligned}
\end{equation*}

推导如下:\\

\begin{equation*}
\begin{aligned}
h'(x) & = \lim_{\Delta x \to 0}\frac{\frac{f(x + \Delta x)}{g(x + \Delta x)} - \frac{f(x)}{g(x)}}{\Delta x}\\
& = \lim_{\Delta x \to 0}\frac{f(x + \Delta x)g(x) - f(x)g(x + \Delta x)}{g(x + \Delta x)g(x)\Delta x}\\
& = \lim_{\Delta x \to 0}\frac{f(x + \Delta x)g(x) - f(x)g(x) + f(x)g(x) - f(x)g(x + \Delta x)}{g(x)g(x)\Delta x}\\
& = \lim_{\Delta x \to 0}\frac{(f(x + \Delta x)- f(x))g(x) - f(x)(g(x + \Delta x) - g(x))}{g^2(x)\Delta x}\\
& = \frac{f'(x)g(x) - f(x)g'(x)}{g^2(x)}
\end{aligned}
\end{equation*}

\subsubsection{反函数的导数}

等于原函数导数的倒数(前提是在给定区间内$f(x)$单调可导且$f'(x) \neq 0$)。推导如下:\\

记原函数为$y = f(x)$,其反函数为$x = f^{-1}(y)$,则:\\

\begin{equation*}
\begin{aligned}
f'(x) = \lim_{\Delta x \to 0}\frac{\Delta y}{\Delta x}\\
(f^{-1}(y))' = \lim_{\Delta y \to 0}\frac{\Delta x}{\Delta y}
\end{aligned}
\end{equation*}

故:\\

\begin{equation*}
(f^{-1}(y))' = \frac{1}{f'(x)}
\end{equation*}

将自变量的表示形式改一下即得:\\

\begin{equation*}
(f^{-1}(x))' = \frac{1}{f'(x)}
\end{equation*}

\subsubsection{复合函数的导数}

若有$y = h(x) = f(g(x))$并令$u = g(x)$,则有:\\

\begin{equation*}
h'(x) = f'(u)g'(x)
\end{equation*}

推导如下:\\

\begin{equation*}
\begin{aligned}
h'(u) &= \frac{dy}{du}\\
\frac{dy}{du}\frac{du}{dx} &= \frac{dy}{du}\frac{du}{dx}\\
\frac{dy}{dx} &= f'(u)g'(x)\\
h'(x) &= f'(u)g'(x)
\end{aligned}
\end{equation*}

注意:求出$f'(u)$值后,还要进一步将$u$转成自变量为$x$的表达式,才能得到最终结果!

\end{document}
